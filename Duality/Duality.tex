\documentclass[10pt,nofootinbib]{revtex4}
%\usepackage{ctex}
\usepackage{amsmath,amssymb,amsfonts,mathrsfs,bm,dsfont}
\usepackage{graphics,color}
%\usepackage{hyperref}

\newcommand*\dd{\mathop{}\!\mathrm{d}}
\newcounter{Claim}[section]
\newenvironment{Claim}[1][]{{\par\normalfont\bfseries \underline{Claim~\stepcounter{Claim}\arabic{Claim}.}~#1~~}}{\par}
\newcounter{Proposition}[section]
\newenvironment{Proposition}[1][]{{\par\normalfont\bfseries \underline{Proposition~\stepcounter{Proposition}\arabic{Proposition}.}~#1~~}}{\par}
\newcounter{Note}[section]
\newenvironment{Note}[1][]{{\par\normalfont\bfseries \underline{Note~\stepcounter{Note}\arabic{Note}.}~#1~~}}{\par}
\newcounter{Lemma}[section]
\newenvironment{Lemma}[1][]{{\par\normalfont\bfseries \underline{Lemma~\stepcounter{Lemma}\arabic{Lemma}.}~#1~~}}{\par}
\newcounter{Corollary}[section]
\newenvironment{Corollary}[1][]{{\par\normalfont\bfseries \underline{Corollary~\stepcounter{Corollary}\arabic{Corollary}.}~#1~~}}{\par}
\newenvironment{Proof}{{\par~{\normalfont\bfseries $\vartriangleright$}~~}}{\hfill $\square$\par\hfill\par} %\par
\newcounter{Def}[section]
\newenvironment{Def}[1][]{{\par\normalfont\bfseries \underline{Definition~\stepcounter{Def}\arabic{Def}.}~#1~~}}{\par}


\def\Re{\mathop{\mathcal{R}e}}
\def\Im{\mathop{\mathcal{I}m}}

\numberwithin{equation}{section}

\begin{document}
\title{Duality in Condensed Matter Physics---From Lattice Models to Particle-Vortex Duality and Beyond}% Force line breaks with \\
%\thanks{This is a reminiscent note for Hubbard-Stratonovich Transformation.}%

\author{Xiaodong Hu}
%\altaffiliation[Also at ]{Boson College}
\email{xiaodong.hu@bc.edu}
\affiliation{Department of Physics, Boston College, MA 02135, USA}

\date{\today}


\begin{abstract}
	This is a research note documenting duality in condensed matter physics.
\end{abstract}
\maketitle
\tableofcontents
\section{Exact Dualities of Lattice Model}
	\subsection{Quantum-Classical Mapping}
		As a warm-up, let us start with the simplest $d=0$ \emph{quantum Ising model}\footnote{Or \emph{transverse field Ising model} if you like. Note that statistical model has \emph{no} time dimensionality.}, namely only one quantum spin with the Hamiltonian
		\begin{equation}\label{1.1.1}
			H_{Q}=-g\sigma^x.
		\end{equation}
		The partition function of \eqref{1.1.1} can be evaluated in path integral formalism by slicing the temperature (imaginary time) into $N$ segments, inserting intermediate states, and {\color{red}let $N$ goes to infinity}
		\begin{align}\label{1.1.2}
			\mathcal{Z}_{Q}&\equiv\mathop{\mathrm{tr}}e^{-\beta H_Q}\equiv\lim_{N\rightarrow\infty}\sum_{s_1,\cdots,s_N}\langle s_N|e^{\Delta\tau g\sigma^x}|s_1\rangle \langle s_1|e^{\Delta\tau g\sigma^x}|s_2\rangle\cdots \langle s_{N-1}|e^{\Delta\tau g\sigma^x}|s_N\rangle\nonumber\\
			&=\lim_{N\rightarrow\infty}\sum_{s_1,\cdots,s_N}\prod_{i=0}^N\langle s_i|1+\Delta\tau g\sigma^x|s_{i+1}\rangle,
		\end{align}
		where $s_0\equiv s_N$ (PBC is applied) and $\Delta\tau\equiv\beta/N\rightarrow0$ such that $\beta$ is fixed.\par
		Equation \eqref{1.1.2} is an reminiscent of the structure of \emph{transferring matrix} in $1$-d \emph{classical} Ising model. In fact, by identifying each intermediate \emph{quantum} state (which is adding by hand in path integral formalism) with \emph{classical} degree of freedom $s_i=\{\pm1\}$ on the physical $N$-site lattice (still with PBC), and ansatzing
		\begin{equation*}
			\langle s_i|1+\Delta\tau g\sigma^x|s_{i+1}\rangle\equiv\langle s_i|Ae^{Bs_i s_{i+1}}|s_{i+1}\rangle,
		\end{equation*}
		or in $\hat{\sigma}^z$ eigenstates
		\begin{equation*}
			\left(\begin{array}{cc}
				1&-ig\Delta\tau\\
				ig\Delta\tau&1
			\end{array}\right)\equiv \left(\begin{array}{cc}
				Ae^B&Ae^{-B}\\Ae^{-B}&Ae^B
			\end{array}\right),
		\end{equation*}
		one immediately have
		\begin{equation}\label{1.1.3}
			A=\sqrt{\Delta \tau g},\quad B=-\dfrac{1}{2}\ln(\Delta\tau g)\rightarrow\infty.
		\end{equation}
		Therefore, the partition function of quantum Ising model can be re-written as
		\begin{equation}\label{1.1.4}
			\mathcal{Z}_{Q}=\lim_{N\rightarrow\infty}A^N\mathop{\mathrm{tr}\exp \left(-\beta H_{c}\right) },
		\end{equation}
		where
		\begin{equation}\label{1.1.5}
			-\beta H_{c}=B\sum_{\langle ij\rangle}\sigma_i^z\sigma_j^z
		\end{equation}
		is exactly the Hamiltonian of $d=1$ classical Ising model.\par

		The above result can be easily generalized to higher dimensions. Hamiltonian of $d>1$ Quantum Ising Model is
		\begin{equation}\label{1.2.1}
			H_q=-J\sum_{\langle ij\rangle}\sigma_i^z\sigma_j^z-g\sum_i\sigma_i^x.
		\end{equation}
		Still clues of the duality theory can be found from its partition function
		\begin{equation}\label{1.2.2}
			\langle s_i|e^{\Delta\tau(J\sum_{\langle ij\rangle}\sigma_i^z\sigma_j^z+g\sum_i\sigma_i^x)+\frac{Jg}{2!}\mathcal{O}(\Delta\tau^2)}|s_j\rangle=\sum_{s_k}\underbrace{\langle s_i|e^{\Delta\tau J\sum_{\langle ij\rangle}\sigma_i^z\sigma_j^z}|s_k\rangle}_{\text{d-dim Classical Ising Model}}\overbrace{\langle s_k|e^{\Delta\tau\sum_i\sigma_i^x}|s_j\rangle}^{\text{0-dim  Quantum Ising Model}},
		\end{equation}
		where B-C-H formula is utilized
		\begin{equation*}
			e^{A}e^B=e^{A+B+\frac12[A,B]+\frac{1}{12}[A,[A,B]]+\cdots}.
		\end{equation*}
		So we say \cite{hsieh2016d} {\color{red}\textbf{Partition function of d-dim quantum statistical model is equivalent to (d+1)-dim classical statistical model}}.
	\subsection{Kramers-Wannier Duality}
	\subsection{$\mathbb{Z}_2$ Gauge Theory}
		$\mathbb{Z}_2$ gauge field describe the fluctuation of gauge freedoms, or visually fluctuation of loops on \emph{infinitely large} square lattice. Spins live on the links of sites, 
		\begin{equation}\label{2.1.1}
			H_{\mathbb{Z}_2}=-K\sum_\square\prod_{\ell\in\square}\sigma_\ell^z-g\sum_\ell\sigma_\ell^x.
		\end{equation}
	\subsection{Jordan-Wigner Transformation to Anyons}
		Consider an \emph{infinite} square lattice, on the links of which spins are placed such that the Hamiltonian

\section{IR Dualities of Continuous Field Theory}
	\subsection{Bosonic Particle/Vortex Duality}
	\subsection{Fermionic Particle/Vortex Duality}
	\subsection{Boson-Fermion Duality and Duality Web}

\section{Application of Duality}

\section{Subtleties on Duality}
	\subsection{Does Duality Keep Entanglement Entropy?}
		Thanks to the deep discussion on \url{https://physics.stackexchange.com/questions/135098}.

\bibliography{hxd}
\bibliographystyle{apsrev} % apsrev is format for PRL of APS
\end{document}