\documentclass[10pt,nofootinbib]{revtex4}
\usepackage{amsmath,amssymb,amsfonts,mathrsfs,bm,dsfont}
\usepackage{graphics,color}

\newcommand*\dd{\mathop{}\!\mathrm{d}}



\begin{document}
\title{Notes on Bosonization}% Force line breaks with \\
\thanks{A footnote to the article title}%

\author{Xiaodong Hu}
%\altaffiliation[Also at ]{Boson College}
\email{xiaodong.hu@bc.edu}
\affiliation{Department of Physics, Boston College}

\date{\today}


\begin{abstract}
Notes of Bosonization.
\end{abstract}
\maketitle
\tableofcontents
\section{1D Electron Gas}
	\subsection{Particularity of 1D Electron Gas}
		Hamiltonian for \emph{compactified}\footnote{To circle $S^1$.} 1DEG is $H=H_0+H_{\text{int}}$ where
		\begin{equation}\label{1.1.1}
			H_0=\sum_{\sigma}\int_0^L\dd\bm{x}\,\psi^\dagger_\sigma(\bm{x}) \left(-\dfrac{\hbar^2}{2m}\dfrac{\partial^2}{\partial x^2}-\mu\right) \psi_\sigma(\bm{x})
		\end{equation}
		and
		\begin{equation}\label{1.1.2}
			H_{\text{int}}=\sum_{\sigma,\sigma'}\int\dd\bm{x}
		\end{equation}
	\subsection{Linearization}
		We are interested in the low energy effective theory of 1DES, in which fermions around FS has the energy
		\begin{equation*}
			\varepsilon(p)\sim(|p|-p_F)v_F.
		\end{equation*}
		In momentum space, this means that only the Fourier compponent of 
		\begin{equation*}
			\psi_\sigma(\bm{x})\equiv\sum\dfrac{\dd p}{2\pi}\psi_\sigma(p)e^{ipx/\hbar}
		\end{equation*}
		near $\pm p_F$ contributes to the description of low-energy states. With the cut off $\Lambda$, one can write
		\begin{align}\label{1.2.1}
			\psi_\sigma(x)&\sim\sum_{-\Lambda}^\Lambda\dfrac{\dd p}{2\pi}\psi_\sigma(-p_F+p)e^{ix(-p_F+p)/\hbar}+\sum_{-\Lambda}^\Lambda\dfrac{\dd p}{2\pi}\psi_\sigma(p_F+p)e^{ix(p_F+p)/\hbar}\nonumber\\
			&=:e^{-ixp_F/\hbar}L_\sigma(x)+e^{ixp_F/\hbar}R_\sigma(x),
		\end{align}
		where
		\begin{equation*}
			L_\sigma(x)\equiv\sum_{-\Lambda}^\Lambda\dfrac{\dd p}{2\pi}\psi_\sigma(p-p_F)e^{ipx/\hbar},\quad R_\sigma(x)\equiv\sum_{-\Lambda}^\Lambda\dfrac{\dd p}{2\pi}\psi_\sigma(p+p_F)e^{ipx/\hbar},
		\end{equation*}
		or in momentum space
		\begin{equation*}
			L_\sigma(p)\equiv\psi_\sigma(p-p_F),\quad R_\sigma(p)\equiv\psi_\sigma(p+p_F).
		\end{equation*}
		Therefore the free Hamiltonian
		\begin{equation}\label{1.2.2}
			H_0=\sum_\sigma\sum\dfrac{\dd p}{2\pi} \varepsilon(p)\psi_\sigma^\dagger(p)\psi_\sigma(p)
		\end{equation}
		can be approximated to narrow integral around the region $p\pm p_F$, i.e.,
		\begin{align}
			H_0&=\sum_\sigma\sum_{-\Lambda}^\Lambda\dfrac{\dd p}{2\pi}\bigg[\varepsilon(p-p_F)\psi_\sigma^\dagger(p-p_F)\psi_\sigma(p-p_F)+\varepsilon(p+p_F)\psi_\sigma^\dagger(p+p_F)\psi_\sigma(p+p_F)\bigg]\nonumber\\
			&=\sum_\sigma\sum_{-\Lambda}^\Lambda\dfrac{\dd p}{2\pi}\,p v_F\bigg(R^\dagger(p)R(p)-L^\dagger(p)L(p)\bigg)
		\end{align}

\section{Bosonization}
	\subsection{Kac-Moody Algebra}
	\subsection{Equivalence of Fermionic and Bosonic Description: Partition Function}

\section{Interactive Terms}

\section{Application on Other Models}

\section{Non-abelian Bosonization}

\section*{Appendix A: Conformal Field Theory}

\end{document}