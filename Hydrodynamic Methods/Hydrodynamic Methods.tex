\documentclass[10pt,nofootinbib,letterpaper]{revtex4}
\usepackage[nocap]{ctex}

%\usepackage{xeCJK}
%\setCJKmainfont{Source Han Sans CN}
%\setCJKmonofont{Source Han Sans CN}
%\setCJKsansfont{Source Han Sans CN}

\usepackage{amsmath,amssymb,amsfonts,mathrsfs,bm,dsfont}
\usepackage{enumerate}
\usepackage{enumitem} % Customize itemize, see https://ctan.org/tex-archive/macros/latex/contrib/enumitem/
\usepackage[all]{xy}
\usepackage[normalem]{ulem}	% delete line
\usepackage{graphics,color}
\usepackage{tikz}
	\usetikzlibrary{calc}
	\usetikzlibrary{decorations.markings}
	\usetikzlibrary{arrows}
	\usetikzlibrary{patterns}
	%\usetikzlibrary{shapes.callouts}
\tikzset{
    level/.style = {
        ultra thick,
        blue,
    },
    connect/.style = {
        dashed,
        red
    },
    label/.style = {
        text width=2cm
    }
}
\usepackage{pgfplots}
%\usepackage[citestyle=authortitle]{biblatex} % able to cite the title, author and year
%\usepackage{hyperref}
\usepackage{feynmp} % feymann diagram
\usepackage{extarrows}

\usepackage[normalem]{ulem} % 文字划掉(横),与 cite 兼容问题,见 https://tex.stackexchange.com/questions/98222/ulem-incompatibility-with-multiple-entries-in-cite

\newcommand*\dd{\mathop{}\!\mathrm{d}}
\newcounter{Claim}[section]
\newenvironment{Claim}[1][]{{\par\normalfont\bfseries \underline{Claim~\stepcounter{Claim}\arabic{Claim}.}~#1~~}}{\par}
\newcounter{Proposition}[section]
\newenvironment{Proposition}[1][]{{\par\normalfont\bfseries \underline{Proposition~\stepcounter{Proposition}\arabic{Proposition}.}~#1~~}}{\par}
\newcounter{Note}[section]
\newenvironment{Note}[1][]{{\par\normalfont\bfseries \underline{Note~\stepcounter{Note}\arabic{Note}.}~#1~~}}{\par}
\newcounter{Lemma}[section]
\newenvironment{Lemma}[1][]{{\par\normalfont\bfseries \underline{Lemma~\stepcounter{Lemma}\arabic{Lemma}.}~#1~~}}{\par}
\newcounter{Corollary}[section]
\newenvironment{Corollary}[1][]{{\par\normalfont\bfseries \underline{Corollary~\stepcounter{Corollary}\arabic{Corollary}.}~#1~~}}{\par}
\newenvironment{Proof}{{\par~{\normalfont\bfseries $\vartriangleright$}~~}}{\hfill $\square$\par\hfill\par} %\par
\newcounter{Def}[section]
\newenvironment{Def}[1][]{{\par\normalfont\bfseries \underline{Definition~\stepcounter{Def}\arabic{Def}.}~#1~~}}{\par}
\newcounter{Assumption}[section]
\newenvironment{Assumption}[1][]{{\par\normalfont\bfseries \underline{Assumption~\stepcounter{Assumption}\arabic{Assumption}.}~#1~~}}{\par}



\allowdisplaybreaks[4] %允许 align 跨页编排


%\def\checkmark{\tikz\fill[scale=0.4](0,.35) -- (.25,0) -- (1,.7) -- (.25,.15) -- cycle;}
%\def\G{\mathcal{G}}
\def\Z{\mathcal{Z}}
\def\H{\mathcal{H}}

\begin{document}
\title{Hydrodynamic Methods in Condensed Matter}
\author{Xiaodong Hu}
%\altaffiliation[Also at ]{Boson College}
\email{xiaodong.hu@bc.edu}
\affiliation{Department of Physics, Boston College}

\date{\today}

\begin{abstract}
	In this note I will review the hydrodynamic methods in condensed matter systems, ranging from non-relativistic symmetry breaking systems like spin waves, to relativistic quantum critical system where symmetry-breaking field (if exists) is not well-defined, and non-relativistic fractional quantum Hall system where the background spacetime is belived to be non-trivial.\par
	%\begin{center}
		\hfill\par
		{\centering\kaishu 云行雨施,品物流形。\\[0.5em]}
	%\end{center}
	\hfill------ 「易经·彖」
\end{abstract}

\maketitle
\tableofcontents

\section{Hydrodynamics in Symmetry-breaking Phase --- Spin Waves}
	\subsection{$S=1/2$ XXZ Model}
		As a practice, let us consider the $S=1/2$ XXZ model in a uniform magnetic field $H_z$
		\begin{equation}\label{1.1.1}
			\H=\H_{XXZ}-H_z\sum_i S_i^z\equiv-\sum_{\langle ij\rangle}J_{ij}^\perp(S_i^xS_j^x+S_i^yS_j^y)-\sum_{\langle ij\rangle}J_{ij}^zS_i^zS_j^z-H_z\sum_i S_i^z.
		\end{equation}
		For simplicity let us narrow our discussion in translation invariant case $J_{ij}^\perp\equiv J_\perp$ and $J_{ij}^z=J_z$. The nature of the ordered phases are determined by the sign\footnote{The sign of $J_\perp$ is unimportant because one can connected two Hamiltonian $H(J_\perp,J_z)$ and $H(-J_\perp,J_z)$ by a unitary transformation $U=\prod_{\text{odd }i} U_i(\hat{z},\pi)$.} and relative magnitude of $J_\perp$ and $J_z$. They are listed as following \cite{chaikin2000principles}:
		\begin{enumerate}[label=\arabic*)]
			\item \emph{Ferromagnetic Heisenberg Model} $J_\perp=J_z=J>0$. The order parameter is a uniform magnetization $\bm{m}$.
			\item \emph{Antiferromagnetic Heisenberg Model} $J_\perp=J_z=J<0$. The order parameter (N\'{e}el oder) is a staggered magnetization $\bm{N}$.
			\item \emph{Planar ferromagnet} $J_\perp>|J_z|>0$. The energy is minimized by parallel alignment of spins in the $xy$-plane. So the order parameter is $\bm{m}=(m_x,m_y)$, or equivalently the complex field $m_\perp(\bm{r})e^{i\varphi(\bm{r})}\equiv m_x(\bm{r})+im_y(\bm{r})$.
			\item \emph{Planar antiferromagnet} $J_\perp<-|J_z|<0$. The energy is minimized by anti-parallel alignment of spins in the $xy$-plane. So the order parameter is the two-component staggered magnetization $\bm{N}\equiv(N_x,N_y)$.
			\item \emph{Ferromagnetic/Antiferromagnetic Ising model} $|J_z|<|J_\perp|$. Since the ground state breaks discrete symmetry for ordered phase, we are not interested in it.
		\end{enumerate}
		Clearly for isotropic models 1) and 2), the entire spin operator $S\equiv\sum_i\bm{S}_i$ commutes with the Hamiltonian so is conserved. While for anisotropic models 3), 4), and 5), only $S^z\equiv\sum_i S_i^z$ is conserved. Let us start the discussion of spin wave theory in planar ferromagnets.

	\subsection{Planar Ferromagnets}
		Operator $S^z$ is conserved in planar ferromagnets, so for the coarse-grained $z$-component magnetization $M_z\equiv\langle S_i^z\rangle$, or magnetization density $m_z(\bm{r})\equiv M_z/V$, we have
		\begin{equation}\label{1.1.2}
			\partial_t m_z+\nabla\cdot\bm{j}^{m_z}=0,
		\end{equation}
		where we introduce the spin current density $\bm{j}^{m_z}$. Another conserved quantity is the total energy density $E\equiv\langle \H_{XXZ}\rangle$, or energy density $\varepsilon$, satisfying
		\begin{equation}\label{1.1.3}
			\partial_t \varepsilon+\nabla\cdot\bm{j}^\varepsilon=0.
		\end{equation}
		\textbf{The conservation of \emph{coarse-grained energy} is definite because it originates from the first component of local diffeomorphism (gauge) transformation on microscopic Hamiltonian, and by Elitzer's theorem gauge symmetry cannot be broken. In constrast, the conservation of \emph{coarse-grained magnetization}, resulting from physical symmetry, may be questionable.}\par
		On the other hand, given $M_z$ and $E$, one still cannot determine the spin configuration of ordered states in planar ferromagnets. In fact, as is mentioned above, one must also specify the direction of alignment of perpendicular magnetization $m_\perp$ (labeled by $\varphi(\bm{r})$), which is locally defined as
		\begin{equation*}
			m_x(\bm{r})+im_y(\bm{r})=m_\perp(\bm{r})e^{i\varphi(\bm{r})}.
		\end{equation*}
		\indent Effective field theory tells us that at physical length scale only countable number of variables are relevant (renormalizable), so in hydrodynamics we have
		\begin{Def}[(Hydro-variables)]
			\textbf{\color{red}In hydrodynamic regime, relevant hydro-variables are those whose long-wavelength variations vary slowly in time compared with the characteristic microscopic relaxation time of the scattering processes}.
		\end{Def}
		In symmetry breaking phase, such variables are \emph{assumed} to be of two classes:
		\begin{itemize}
			\item densities of conserved variables (here is $m_z(\bm{r})$ and $\varepsilon(\bm{r})$);
			\item symmetry-breaking elastic variables (here is $\varphi(\bm{r})$).
		\end{itemize}
		\par The fundamental assumption of hydrodynamics is
		\begin{Assumption}
			\textbf{Each point of the system reaches thermodynamic equilibrium at each instant of time, so that the system is \emph{completely} determined by hydro-variables, even though they vary in time}. 
		\end{Assumption}
		\noindent Particularly, the above unspecified current $\bm{j}^e$ and $\bm{j}^{m_z}$ should be functional of these hydro-variables. Ditto for the symmetry-breaking field if we introduce a scalar valued functional $\psi$ of hydro-variables and \emph{force} it to satisfy
		\begin{equation}\label{1.1.4}
			\partial_t \varphi(\bm{r})+\psi(\bm{r})=0.
		\end{equation}
		Noting that it is the fluctuation of $\nabla\varphi$ rather than $\varphi$ that contributes to physical observables in the same order as $\varepsilon$ and $m_z$. So we will take $\bm{v}(\bm{r})\equiv\nabla\varphi(\bm{r})$ as a replacement of fundamental hydro-variables ($\bm{v}$ is in anolagous of superfluid velocity $\bm{v}_s$ in liquid helium), satisfying
		\begin{equation}\label{1.1.5}
			\partial_t\bm{v}(\bm{r})+\nabla\psi(\bm{r})=0.
		\end{equation}
		Equation \eqref{1.1.2}, \eqref{1.1.3}, and \eqref{1.1.5} consist of the complete ``conservation law'' of our system.

	\subsection{Constitutive Relations}
		Equations relating coarse-grained current densities with hydro-variables are called \emph{constitutive relations}. Introducing the conjugate field $\bm{x}$ of $\bm{v}$, and $h$ of $m_z$, as generalized forces, the first law of thermodynamics (in true thermodynamic eqiuilibrium) tells
		\begin{equation}\label{2.0.1}
			T\dd s\equiv\dd \varepsilon-h\dd m_z-\bm{x}\cdot\dd\bm{v},
		\end{equation}
		where $h\equiv-T \partial s/\partial m_z|_S$ and $\bm{x}\equiv-T\partial s/\partial \bm{v}|_S$. 

		\subsubsection{Zeroth Order (Non-dissipative)}
			In this subsection, we will \textbf{use the conservation of local entropy to obtain the \emph{non-dissipative} part of ``current operators'' introduced above}. \par
			To the zeroth order of fluctuations, i.e., no spatial derivatives of hydro-variables, the most general form of current operator we can write is
			\begin{equation}\label{2.1.1}
				\bm{j}^\varepsilon\equiv A(\bm{r})\bm{v}(\bm{r}),\quad \bm{j}^{m_z}\equiv B(\bm{r})\bm{v}(\bm{r}),\quad \psi\equiv C(\bm{r}),
			\end{equation}
			where coefficients $A,B$ and $C$ are scalar functional of $\varepsilon(\bm{r})$ and $m_z(\bm{r})$ waiting to be determined. In general, coefficients $A$ and $B$ may be even chosen as rank-$2$ tensors of functionals of $\varepsilon$ and $m_z$, such that the two vector-valued current operators $\bm{j}^\varepsilon$ and $\bm{j}^{m_z}$ are not in the same direction as $\bm{v}$. {\color{red}But here we shall assume lattice \emph{cubic symmetry} \cite{halperin1969hydrodynamic} to forbid such possibility}. Simiarly, one can contract vector $\bm{v}$ with a rank-2 tensor to form a scalar functional allowed in the expression of $\psi$. {\color{red}But we shall assume \emph{reflection symmetry} to forbid such terms for convenience}.\par

			Fortunately, as we will see below, $C(\bm{r})$ can be obtained in advance without entropy argument \cite{chaikin2000principles,halperin1969hydrodynamic}.\par
			Our aim (spin wave mode) is to find a \emph{stationary state} (in comparison with true thermodynamic equilibrium) with a non-vanishing constant precession rate from \eqref{1.1.4} that $\dd\varphi/\dd t=-\psi_0$, which is determined by conserved hydro-variables $\varepsilon,m_z$, and $\bm{v}$ if it is nonzero.\par
			Let us assume the existence of such stationary state \emph{without} external magnetic field, and turn on $H_z$. The time evolution of $m_\perp(t,H_z)\equiv\langle S_i^+(t)\rangle/V$ is now dominated by the Hamiltonian $\mathcal{H}\equiv\mathcal{H}_{XXZ}+\mathcal{H}_{ext}$. And because $\mathcal{H}_{ext}\equiv-H_z\sum_i S_i^z$ commutes with $\mathcal{H}_{XXZ}$,
			\begin{equation*}
				m_z(t,H_z)=\dfrac{1}{V}\bigg\langle e^{i\mathcal{H}t}S_i^+(0)e^{-i\mathcal{H}t}\bigg\rangle=\dfrac{1}{V}\bigg\langle e^{i\mathcal{H}_{XXZ}t}e^{-iH_z\sum_iS_i^zt}S_i^+(0)e^{iH_z\sum_iS_i^zt}e^{-i\mathcal{H}_{XXZ}t}\bigg\rangle=e^{-iH_zt}m_z(t,0)
			\end{equation*}
			gives
			\begin{equation*}
				\dfrac{\dd\varphi}{\dd t}=-(\psi_0(\varepsilon,m)+H_z).
			\end{equation*}
			However, in \emph{true} thermodynamic equilibrium (with non-vanishing $H_z$), where the system energy $E-M_z H_z$ is minimized so that $M_z$ (now with non-vanishing $H_z$) obeys
			\begin{equation}\label{2.1.2}
				\dfrac{\partial E}{\partial M_z}=\dfrac{\partial \varepsilon}{\partial m_z}=H_z,
			\end{equation}
			the procession rate $\dd\varphi/\dd t$ must be zero, otherwise the rotating $m_\perp$ will radiate and lose energy \cite{halperin1969hydrodynamic}. Thus $\psi_0$ coincides with $H_z$ in \emph{true} equilibrium. On the other hand, the fisrt law of thermodynamics \eqref{2.0.1} tells us the left hand side of \eqref{2.1.2} is nothing but conjugate field $h(\varepsilon,m_z)\equiv\frac{\partial \varepsilon}{\partial m_z}$. Therefore, we conclude for general \emph{non-dissipative} stationary state 
			\begin{equation}\label{2.1.3}
				\dfrac{\dd\varphi}{\dd t}=-(h-H_z),
			\end{equation}
			or
			\begin{equation}\label{2.1.4}
				C(\bm{r})=H_z-h(\varepsilon(\bm{r}),m_z(\bm{r})).
			\end{equation}
			\indent The left work is to determine $A(\bm{r})$ and $B(\bm{r})$ in \eqref{2.1.1} with the help of entropy conservation (at zeroth order).\par
			Hydrodynamic assumption ensures local entropy density to be the functional of hydro-variables $s=s(\varepsilon,m_z,\bm{v})$. And at each slice of time, each point of the system is assumed to reach local thermodynamic eqiuilibrium. Therefore we can expand entropy density around its equilibrium value $s_0$ (which must be the maximum of $s$ due to the second law of thermodynamics) that 
			\begin{equation}\label{2.1.5}
				s\simeq s_0(\varepsilon)-\dfrac{1}{2T}\chi_s^{-1}m_z^2-\dfrac{\rho_s}{2T}v^2,
			\end{equation}
			where temperature $T^{-1}\equiv\partial s_0/\partial\varepsilon$ is well-defined in equilibrium and inserted by convention. Magnetic \emph{static} suseptibility $\chi_s$ (we will see latter why such coefficient is the suseptibility) and the magnetic version of ``superfluid density'' $\rho_s$ may also be functional of hydro-variables, but up to the second order of disturbance, we can safely treat them as constants\footnote{Apparently $\chi_s$ and $\rho_s$ must be nonnegative for stability of the system.}. Therefore the two conjugate fields we introduce above takes the form of $h\equiv-T \partial s/\partial m_z=\chi_s^{-1}m_z$ and $\bm{x}\equiv-T\partial s/\partial \bm{v}=\rho_s\bm{v}$.\par
			Taking time derivatives\footnote{Strickly speaking, we are comparing one-forms on distinct points of base space manifold connecting by the vector field $\partial/\partial t$. So it is Lie derivative.} of the first law of thermodynamics and substituting conservation laws \eqref{1.1.3}, \eqref{1.1.4}, \eqref{1.1.5} and the zeroth order of non-dissipative currents \eqref{2.1.1} and \eqref{2.1.4}, we have
			\begin{align}\label{2.1.6}
				T\dfrac{\partial s}{\partial t}&=\dfrac{\partial \varepsilon}{\partial t}-h\dfrac{\partial m_z}{\partial t}-\bm{x}\cdot\dfrac{\partial \bm{v}}{\partial t}=-\nabla\cdot\bm{j}^\varepsilon+h\nabla\cdot\bm{j}^{m_z}-\bm{x}\cdot\nabla(h-H_z)\nonumber\\
				&=-\nabla\cdot(A\bm{v})+h\nabla\cdot(B\bm{v})-\bm{x}\cdot\nabla h\nonumber\\
				&=-\nabla\cdot\bigg((A-Bh)\bm{v}\bigg)-(B\bm{v}+\bm{x})\cdot\nabla h.
			\end{align}
			Writing the \emph{heat current} $\bm{Q}\equiv (A-Bh)\bm{v}$ and using the identity
			\begin{equation*}
				\nabla\cdot\bm{Q}\equiv T\nabla\cdot \left(\dfrac{\bm{Q}}{T}\right)+\bm{Q}\cdot \left(\dfrac{\nabla T}{T}\right),  
			\end{equation*}
			Equation \eqref{2.1.6} becomes
			\begin{equation}\label{2.1.7}
				T\dfrac{\dd s}{\dd t}\equiv T \left(\dfrac{\partial s}{\partial t}+\nabla\cdot \left(\dfrac{\bm{Q}}{T}\right) \right)=-\bm{Q}\cdot \left(\dfrac{\nabla T}{T}\right)-(B\bm{v}+\bm{x})\cdot\nabla h
			\end{equation}
			if we identify $\bm{Q}$ as the \emph{entropy current density}. The entropy production \eqref{2.1.7} must be non-negative. But in the absence of dissipation (reversible processes), the entropy remains constant $\frac{\dd s}{\dd t}=0$. Thus
			\begin{equation}\label{2.1.8}
				\begin{cases}
					\bm{Q}\equiv (A-Bh)\bm{v}=0\\
					B\bm{v}+\bm{x}=0
				\end{cases}\implies
				\begin{cases}
					A\bm{v}=-\rho_s\bm{x}\\
					B\bm{v}=-\bm{x}
				\end{cases}.
			\end{equation}
			Inserting the expression of $h$ and $\bm{v}$, finally we come to the \emph{non-dissipative} linearized equation of motion up to the zeroth order of gradients of hydro-variables
			\begin{align}
				\partial_t\varepsilon&=\dfrac{\rho_s}{\chi_s}\nabla\cdot(m_z\bm{v})\label{2.1.9}\\
				\partial_t m_z&=\rho_s\nabla\cdot\bm{v}\label{2.1.10}\\
				\partial_t \bm{v}&=\dfrac{1}{\chi_s}\nabla m_z\label{2.1.11}.
			\end{align}
			Taking the time derivative of \eqref{2.1.10} and substituting \eqref{2.1.11}, one immediately have
			\begin{equation}\label{2.1.12}
				\dfrac{\partial^2 m_z}{\partial t^2}=-\dfrac{\rho_s}{\chi_s}\nabla^2m_z,
			\end{equation}
			which predicts a \emph{undamped spin-wave mode} $\omega=\pm ck$ with $c\equiv(\rho_s/\chi_s)^{1/2}$.

		\subsubsection{First Order (Dissipative)}
			To the first order gradients of hydro-variables, howevere, we will use {\bf the positivity of local entropy production} to constrain the form of \emph{dissipative} part of currents
			\begin{equation}\label{2.2.1}
				\bm{j}^\varepsilon\equiv-\rho_s\bm{x}+{\bm{j}^\varepsilon}',\quad\bm{j}^{m_z}\equiv-\bm{x}+{\bm{j}^{m_z}}',\quad \psi\equiv H_z-h+\psi'.
			\end{equation}
			Here the zeroth-order result has been used.	Similar to the precedure we have done above, introducing the first-order \emph{heat current} $\bm{Q'}\equiv{\bm{j}^\varepsilon}'-h{\bm{j}^{m_z}}'+\rho_s\bm{v}\psi'$, the local entropy production now takes the form of
			\begin{equation}\label{2.2.2}
				T\dfrac{\dd s}{\dd t}\equiv T \left(\dfrac{\partial s}{\partial t}+\nabla\cdot\bm{Q'}\right)=-\bm{Q'}\cdot\left(\dfrac{\nabla T}{T}\right)-{\bm{j}^{m_z}}'\cdot\nabla h-\psi'\nabla\cdot\bm{x}.
			\end{equation}
			The most general constitutive relations up to the \emph{first order} we can construct are
			\begin{align*}
				\bm{Q}'&=-K_{11}\nabla T-K_{12}\nabla h\\
				{\bm{j}^{m_z}}'&=-K_{21}\nabla T-K_{22}\nabla h\\
				\psi'&=K_{33}\rho_s\nabla\cdot\bm{v}.
			\end{align*}
			In general both $\bm{Q}'$ and ${\bm{j}^{m_z}}'$ may also contain terms proportional to spacial fluctuation of $\bm{v}$ like $\nabla\times\bm{v}$. But the definition $\bm{v}\equiv\nabla\varphi$ forbids such terms. There may also be possibility that $\bm{Q'},{\bm{j}^{m_z}}'$ be propotional to the contraction of the gradient of $T$ and $h$ with a rank-$2$ tensor, but the {\color{red}\textbf{cubic symmetry}} prohibits such terms. In addition, $\psi'$ may also be propotional to terms like $\bm{n}\cdot\nabla T$ and $\bm{n}\cdot\nabla h$ with some specific direction vector $\bm{n}$, but again the {\color{red}\textbf{cubic symmetry}} prohibits them.\par
			Moreover, since we are interested in the fluctuation near true thermodynamic equilibrium \emph{without}\footnote{This will no do harm with our previous discussion to the zeroth order.} external magnetic fields $H_z$ (so that the magnetization $M_z=0$ and time-reversal symmetry is kept), \textbf{\color{red} whatever the direction of time arrow is, the entropy production $\frac{\dd s}{\dd t}$ by assumption, will always keep positive}. Therefore \textbf{all term on the RHS of \eqref{2.2.2} must be even under time-reversal transformation}. In light of the fact that $T$ is time-reversal even and $h$ is time-reversal odd, both off-diagonal couplings $K_{12}$ and $K_{21}$ should be zero, leaving
			\begin{equation}\label{2.2.3}
				\bm{Q}'=-\kappa\nabla T,\quad {\bm{j}^{m_z}}'=-K_{22}\nabla h,\quad \psi'=-K_{33}\rho_s\nabla\cdot\bm{v},
			\end{equation}
			where we interprete $K_{11}$ as the \emph{thermal conductivity} $\kappa$.\par
			Vanishing of $M_z$ helps to make equation \eqref{2.2.3} close. In fact, by the first law of thermodynamics
			\begin{equation*}
				\dd T=C^{-1}\dd\varepsilon
			\end{equation*}
			if we introduce specific heat $C$. Inserting back the expression of dissipative currents to conservation laws, we come to the \emph{dissipative} linearized\footnote{By linearity, we mean the equation of motion is expanded to the first order of hydro-variables.} equation of motion up to the first-order fluctuation of hydro-variables
			\begin{align}
				\partial_t \varepsilon&=\nabla\cdot\bigg(\kappa\nabla T+K_{22}h\nabla h-K_{33}\rho_s^2\bm{v}\nabla\cdot\bm{v}\bigg)\simeq\kappa C^{-1}\nabla^2\varepsilon\label{2.2.4}\\
				\partial_t m_z&=\rho_s\nabla\cdot\bm{v}+\chi_s^{-1}K_{22}\nabla^2m_z\label{2.2.5}\\
				\partial_t\bm{v}&=\chi_s^{-1}\nabla m_z+K_{33}\rho_s\nabla(\nabla\cdot\bm{v})\label{2.2.6}.
			\end{align}
			Clearly solution of \eqref{2.2.4} yields a \emph{diffusive heat mode}
			\begin{equation}\label{2.2.7}
				\omega_\varepsilon(\bm{k})=-i\kappa C^{-1}k^2=:-iD_Tk^2.
			\end{equation}
			Another \emph{damped spin-wave mode} from coupled equations \eqref{2.2.5} and \eqref{2.2.6} can be obtained by diagnolizing the coefficients matrix in momentum space (by choosing $\bm{k}$ along one specific direction)
			\begin{equation*}
				\mathcal{M}_{m_z,\bm{v}}=\left(\begin{array}{cc}
					\chi^{-1}K_{22}k^2 & \rho_s k \\
					\chi^{-1}k & \rho_s K_{33} k^2
				\end{array}\right)
			\end{equation*}
			and expanding up to $k^2$, giving
			\begin{equation}\label{2.2.8}
				\omega_{\pm}(\bm{k})=\pm ck-\dfrac{1}{2}iDk^2,
			\end{equation}
			where $D\equiv\chi_s^{-1}K_{22}+\rho_s K_{33}$.
	
	\subsection{Hydrodynamic Response Functions}
		In this section, we try to find the dynamic suseptibility/correlation function of hydro-variables from the equation of motion from \eqref{2.2.4} to \eqref{2.2.6}. A good review is given in \cite{kovtun2012lectures}. I just use Kovtun's result.\par
		Diffusion equation \eqref{2.2.4} is apparently decoupled from the other two hydro-variables. So let us start our discussion with the susceptibility $\chi_{\varepsilon \varepsilon}$. But \textbf{to discuss linear response, one has to specify the source of perturbation in advance}. A disturbance in energy, $\delta \varepsilon$, can be caused by the disturbance of temperature $-\beta\mathcal{H}\mapsto-(\beta+\delta\beta)\mathcal{H}$, or more explicitly,
		\begin{equation}\label{2.3.1}
			\delta\mathcal{H}=-\int\dd\bm{x}\,\dfrac{\delta T}{T}\delta\varepsilon(t,\bm{x}).
		\end{equation}
		So the \emph{static} susceptibility
		\begin{equation}\label{2.3.2}
			\chi_{\varepsilon \varepsilon}\equiv\dfrac{\partial\delta\varepsilon}{\partial\left(\dfrac{\delta T}{T}\right)}=CT.
		\end{equation}
		So in the long wavelength limit the retarded Green function \cite{kovtun2012lectures,chaikin2000principles}
		\begin{equation}\label{2.3.3}
			G_{\varepsilon\varepsilon}^R(z,\bm{k})\equiv\dfrac{\chi_{\varepsilon \varepsilon}D_T\bm{k}^2}{iz-D_T\bm{k}^2},
		\end{equation}
		or in hydrodynamic regime where $\beta\omega\ll1$ so that
		\begin{equation}\label{2.3.4}
			G_{\varepsilon \varepsilon}(\omega,\bm{k})=-\dfrac{2k_BT}{\omega}\mathop{\mathrm{Im}}G^R_{\varepsilon \varepsilon}=2k_BCT^2\dfrac{D_T\bm{k}^2}{\omega^2+(D_T\bm{k}^2)^2}.
		\end{equation}
		\indent Things will not change a lot if we include the other two equation of motion \eqref{2.2.5} and \eqref{2.2.6}. In fact, we only need to extend the above formula to matrix form. Clearly the disturbance of source can be read from the first law of thermodynamics \eqref{2.0.1}
		\begin{equation}\label{2.3.5}
			\delta\mathcal{H}=-\int\dd\bm{x}\bigg(\delta h\cdot\delta m_z+\delta\bm{x}\cdot\delta\bm{v}\bigg).
		\end{equation}
		So correspondingly under the basis of $(m_z,\bm{v})^T$ we have the static susceptibility
		\begin{equation}\label{2.3.6}
			\chi\equiv\dfrac{\partial \lambda_a}{\partial \varphi_b}=\left(\begin{array}{cc}
				\chi_s & \\ & \rho_s
			\end{array}\right),
		\end{equation}
		where $\lambda_a$ is the source and $\varphi_b$ is hydro-variable. (As a contrast, Hohenberg and Halpersin alter the hydro-variable $\bm{v}$ to $m_y$ in their original paper \cite{halperin1969hydrodynamic}). Retarded Green function can also be obtained by Laplace transformation and the connection with dynamic susceptibility \cite{kovtun2012lectures} that
		\begin{align}
			G^R(\omega,\bm{k})&=-(\mathds{1}+i\omega K^{-1})\chi,\label{2.3.7}\\
			&=\dfrac{1}{\chi_s^{-1}K_{22}\rho_s K_{33}k^4-i(\chi_s^{-1}K_{22}+\rho_sK_{33})\omega k^2-\rho_s\chi_s^{-1}k^2-\omega^2}\nonumber\\
			&\qquad\times \left(\begin{array}{cc}
				k^2\bigg((k^2 K_{22}
				K_{33}-1) \rho_s+i K_{22}\omega\bigg) & ik \rho_s^2\omega\\
				i k\omega & k^2\rho_s\bigg(\chi_s^{-1}\rho_s+i\rho_sK_{33}\omega-\chi_s^{-1}K_{22}\rho_sK_{33} k^2\bigg)\\
				\end{array}\right).\label{2.3.8}
		\end{align}
		where $K_{ab}\equiv -i\omega\delta_{ab}+\mathcal{M}_{ab}$. The physical meaning of denomenator can be seen by solving the singularity up to the second order of $k$, in fact
		\begin{equation*}
			\chi_s^{-1}K_{22}\rho_s K_{33}k^4-i(\chi_s^{-1}K_{22}+\rho_sK_{33})\omega k^2-\rho_s\chi_s^{-1}k^2-\omega^2\equiv(\omega-\omega_-)(\omega-\omega_+)+\mathcal{O}(k^3).
		\end{equation*}
		And the full Green function can be read from
		\begin{align}
			G_{AB}&\equiv-\dfrac{2k_BT}{\omega}\mathop{\mathrm{Im}}G^R_{AB}.
		\end{align}
	\subsection{Planar Antiferromagnets}
		Ground states of planar antiferromagnets are anti-parallel spin configuration confined in $x$-$y$ plane, altering the sign site by site. Dividing the lattice into two sublattice $A$ and $B$ and defining a new spin-like variable (in $x$-$y$ plane)
		\begin{equation*}
			\bm{Q}_i=\eta_{i}\bm{S}_i,
		\end{equation*}
		where $\eta_i=+1$ on sublattice $A$ while $-1$ on sublattice $B$, then clearly $\{Q_i^x,Q_i^y,S_i^z\}$ obeys the same commutation relation. More importantly, in terms of the new group of dynamical variables, the XXZ Hamiltonian can be re-written as
		\begin{equation}\label{2.4.1}
			\mathcal{H}_{XXZ}=-\sum_{\langle ij\rangle}J_z S_i^z S_j^z-\sum_{\langle ij\rangle}\eta_i\eta_j J_\perp(Q_i^x Q_j^x+Q_i^y Q_j^y)+\mathcal{H}_{ext}.
		\end{equation}
		Recognizing $\eta_i\eta_j J_\perp$ with a new \emph{ferromagnetic coupling} $\widetilde{J}_\perp$, then clealy we come back to the same Hamiltonian in \textbf{ferromagnets}. Moreover, defining the \emph{slowly varying} staggered magnetization density $\bm{n}(\bm{r})$ similar to $m(\bm{r})$
		\begin{equation}\label{2.4.2}
			\bm{n}(\bm{r})\equiv\left\langle\sum_i\bm{Q}_i\delta(\bm{r}-\bm{r}_i)\right\rangle,
		\end{equation}
		whose orientation gives a local symmetry-breaking field $\varphi(\bm{r})$  and corresponding vector field $\widetilde{\bm{v}}\equiv\nabla\varphi$ again
		\begin{equation}\label{2.4.3}
			n_x(\bm{r})+in_y(\bm{r})\equiv n_\perp(\bm{r})e^{i\varphi(\bm{r})}.
		\end{equation}
		So with a new group of hydro-variables $\{m_z,\varepsilon,\widetilde{\bm{v}}\}$, \textbf{\color{red}all previous hydrodynamic treatments of spin-wave in planar ferromagnets and the equation motion applies here}.

	\subsection{Isotropic Antiferromagnets}
		However, things changed in isotropic Heisenberg models. For isotropic antiferromagnets, all components (instead of merely $m_z$ in planar magnets) of the magnetization are conserved
		\begin{equation}\label{2.5.1}
			\dfrac{\partial m_i}{\partial t}+\nabla\cdot\bm{j}^{m_i}=0.
		\end{equation}
		That is, there are four conserved variables $\{m_x,m_y,m_z,\varepsilon\}$ in all.\par
		The orientation of staggered magnetization $\bm{n}$, as is introduced before, will give rise to the slowly-varying symmetry-breaking variables. Suppose in ordered phase $\bm{n}=\hat{z}$ in equilibrium, then \textbf{small fluctuation of staggered magnetization is obviously in $x$-$y$ plane $\delta\bm{n}\simeq(\delta n_x,\delta n_y,0)=(\delta\theta_y,-\delta\theta_x,0)$ and dominated by \emph{two} independent symmetry-breaking hydrovariables $\theta_x$ and $\theta_y$} (As a comparison, for planar magnets there is only one parameter controlling the orientation of planar $\bm{n}$). And it is the \emph{spatial} flucuation of $\delta\bm{n}$ that contributes to physical observable. So we work in the vector fields for each component one more time,
		\begin{equation*}
			\bm{v}_x\equiv\nabla\theta_y,\quad \bm{v}_y\equiv-\nabla\theta_x,
		\end{equation*}
		satisfying
		\begin{equation}\label{2.5.2}
			\partial_t\bm{v}_i+\nabla\psi_i=0
		\end{equation}
		if we introduce a functional $\psi_i$ of hydro-variables.\par
		Therefore, for antiferromagnets the first law of thermodynamics is
		\begin{equation}\label{2.5.3}
			T\dd s=\dd \varepsilon-\bm{h}\cdot\dd\bm{m}-\bm{x}_x\cdot\dd\bm{v}_x-\bm{x}_y\cdot\dd\bm{v}_y,
		\end{equation}
		and small fluctuation of entropy takes the form of\footnote{The reason why the coefficients of $m_x$ and $m_y$ are the same comes from the cubic symmetry of our system.}
		\begin{equation}\label{2.5.4}
			s=s_0-\dfrac{1}{2T}\bigg[\chi_\parallel^{-1}m_z^2+\chi_\perp^{-1}(m_x^2+m_y^2)\bigg]-\dfrac{\rho_s}{2T}\bigg[\bm{v}_x^2+\bm{v}_y^2\bigg],
		\end{equation}
		where
		\begin{equation}\label{2.5.5}
			h_i=-T\dfrac{\partial s}{\partial m_i}=\chi_\perp^{-1}m_i,\quad \bm{x}_i\equiv-T\dfrac{\partial s}{\partial \bm{v}_i}=\rho_s\bm{v}_i.
		\end{equation}
		Following almost the same analysis on the constraint of constitutive relation with the help of symmetries, one finally gets
		\begin{align}
			\dfrac{\partial \varepsilon}{\partial t}&=\kappa\nabla^2T=\kappa C^{-1}\nabla^2 \varepsilon,\label{2.5.6}\\
			\dfrac{\partial m_z}{\partial t}&=\chi^{-1}K_\parallel\nabla^2 m_z,\label{2.5.7}\\
			\dfrac{\partial m_i}{\partial t}&=\rho_s\nabla\cdot\bm{v}_i+\chi^{-1}K_\perp\nabla^2 m_i,\quad i=x,y,\label{2.5.8}\\
			\dfrac{\partial \bm{v}_i}{\partial t}&=\chi^{-1}\nabla m_i+K\rho_s\nabla(\nabla\cdot\bm{v}_i),\quad i=x,y,\label{2.5.9}
		\end{align}
		up to first-order dissipative equation of motion, where $\kappa,K_\parallel,K_\perp$, and $K$ are expanding constants for constitutive relations. Clearly, \eqref{2.5.6} and \eqref{2.5.7} give two diffusive modes, while \eqref{2.5.8} and \eqref{2.5.9} give \emph{damped spin-wave modes}
		\begin{equation}\label{2.5.10}
			\omega_\pm=\pm(\rho_s/\chi_\perp)^{1/2}k+\dfrac{i}{2}(K_\perp\chi_\perp^{-1}+K\rho_s)k^2.
		\end{equation}
		Correlation function can also be obtained from the above four equations of motion.

	\subsection{Isotropic Ferromagnets}
		Things become more subtle when we come to isotropic ferromagnets. This time, \textbf{$m_z$ plays the role as \emph{both} a slowly-varying conserved variable and broken-symmetry variable}. Since hydro-variables CANNOT be count twice, our first law of thermodynamics reduces to
		\begin{equation}\label{2.6.1}
			T\dd s=\dd \varepsilon-\bm{h}\cdot\dd\bm{m}
		\end{equation}
		in both disordered and ordered phases.\par
		Similar to previous analysis in isotropic antiferromagnets, given a equilibrium orientation of uniform magnetization $\bm{m}=\hat{z}$, the deviation of $\bm{m}$ is in $x$-$y$ plane. But in accordance with the first law of thermodynamics \eqref{2.6.1}, we still use $m_x$ and $m_y$ rather than their orientation parameter to characterize the slowly varying symmetry-breaking fields. Thus the entropy for small fluctuation can be expanded as
		\begin{equation}\label{2.6.2}
			s=s_0-\dfrac{1}{2T}\chi_\parallel^{-1}m_z^2-\dfrac{\rho_s}{2T M_0^2}\bigg[(\nabla m_x)^2+(\nabla m_y)^2\bigg],
		\end{equation}
		where $M_0$ is added just for dimensionality. So we know
		\begin{equation*}
			h_z=\chi_\parallel^{-1}m_z,\quad h_i=-\dfrac{\rho_s}{M_0^2}\nabla^2m_i.
		\end{equation*}
		Again the non-dissipative part of equation
		\begin{equation}\label{2.6.3}
			\partial_t m_i+\nabla\cdot\bm{j}^{m_i}=0
		\end{equation}
		can be obtained direct from the evolution of steady state 
		\begin{equation}\label{2.6.4}
			\partial_tm_i=-(\bm{m}\times h)_i
		\end{equation}
		since the equation of motion for spin is $\partial \bm{S}^\alpha/\partial_t=\bm{S}^\alpha\times H_z$ when there is external magnetic fields. For the full dissipative equation of motion up to the first-order fluctuation of hydro-variables, the equations of motion are given in \cite{chaikin2000principles} that
		\begin{align}
			\partial_t \varepsilon&=\kappa C^{-1}\nabla^2 \varepsilon,\label{2.6.5}\\
			\partial_t m_z&=\Gamma\chi_\parallel^{-1}\nabla^2 m_z,\label{2.6.6}\\
			\partial_t m_x&=-\dfrac{\rho_s}{M_0^2}\nabla^2 m_y-\Gamma\dfrac{\rho_s}{M_0^2}\nabla^4 m_x,\label{2.6.7}\\
			\partial_t m_y&=\dfrac{\rho_s}{M_0^2}\nabla^2 m_x-\Gamma\dfrac{\rho_s}{M_0^2}\nabla^4 m_y.\label{2.6.8}
		\end{align}
		Clearly \eqref{2.6.5} and \eqref{2.6.6} yield two diffusive modes, while \eqref{2.6.7} and \eqref{2.6.8} give rise to \emph{damped spin waves}
		\begin{equation}\label{2.6.9}
			\omega_\pm=\pm(\rho_s/M_0^2)^{1/2}k^2+i \widetilde{D}k^4, 
		\end{equation}
		where $\widetilde{D}\equiv\Gamma\rho_s/M_0^2$.

<<<<<<< HEAD

\section{Transport of Weyl Semimetals}
	In this section, we focus on \emph{relativistic} transport in Weyl semimetals. The system of Dirac fermions will couple with external electro-magnetic fields for linear response study, so that the action in general can be written as $S[\psi,A_\mu,a_\mu,g_{\mu\nu},\phi]$, where $A_\mu$ is the \emph{external} (\emph{non-dynamic}) gauge field, $a_\mu$ is the \emph{internal} gauge field that has nothing to do with transport currents, $g_{\mu\nu}$ is the metric tensor for the definition of symmetric energy-momentum tensor, and $\phi$ are the other internal matter fields.\par
	Unlike the unsystematic operations in spin waves or superfluids, modern treatments on conservation laws are tighly connected with the microscopic lagrangian so that the coarse-grained currents and energy-momentum tensor are (for simplicity we will denote the Euclidean action as $S=S[\psi,A_\mu,g_{\mu\nu}]$)
	\begin{equation}\label{3.1.1}
		\langle J^\mu\rangle\equiv\dfrac{1}{Z}\int\mathcal{D}(\bar{\psi},\psi)\,\dfrac{\delta S}{\delta A_\mu}e^{-S}\equiv\dfrac{\delta W}{\delta A_\mu},\quad \langle T^{\mu\nu} \rangle\equiv\dfrac{2}{\sqrt{g}}\dfrac{\delta W}{\delta g_{\mu\nu}},
	\end{equation}
	where the generating functional $W$ satisfies
	\begin{equation*}
		e^{W[g,A]}\equiv Z[g,A]\equiv\int\mathcal{D}(\bar{\psi},\psi)\,e^{-S[\psi,g,A]}.
	\end{equation*}

\section{Transport Near Quantum Criticality}
	In this section, we will focus on \emph{Lorentz-invariant} quantum critical points of \emph{superfluid-mott insulator phase transition} in the \emph{hydrodynamics region} that external electromagnetic frequencies satisfy $\hbar\omega\ll k_B T$. This condition is widely (actually, almost all) performed in experiments but mismatched in the assumption of many theoretical calculation as well as numerical simulations in, for example, DC conductivity near superfluid-insulator phase transition \cite{damle1997nonzero}.\par
=======
\section{Transport Near Quantum Criticality}
	In this section, we will focus on \emph{Lorentz-invariant} quantum critical points of superfluid-insulator phase transition in the \emph{hydrodynamics limit} that external electromagnetic frequencies satisfy $\hbar\omega\ll k_B T$. This condition is widely (actually, almost all) performed in experiments but mismatched in the assumption of many theoretical calculation as well as numerical simulations in, for example, DC conductivity near superfluid-insulator phase transition \cite{damle1997nonzero}.\par
>>>>>>> 80311c9c53d78486822103cfeb432d0edc2c369f
	Unlike in the theory of classical finite temperature critical points, where $k_bT/\hbar$ constitutes an intrinsic relaxation time for the fluctuation of symmetry-breaking fields (as hydro-variables), we are interested in much lower frequency scales so that the dynamics of order parameter can be ignored \cite{hartnoll2007theory}.

	\subsection{General Setup}

		For transport study our system subjects to external magnetic field. Invariance of the general action under \emph{gauge transformation} $A_\mu\mapsto A_\mu+\partial_\mu f$ and \emph{diffeomorphism transformation} $x^\mu\mapsto x_\mu+\xi_\mu$ gives\footnote{See, for example, section II D of \cite{herzog2009lectures}. The result can also be proved holographically, see \cite{lindgren2015holographic}.}
		\begin{align}\label{5.1.1}
			\partial_\mu J^\mu&=0,\\
			\partial_\mu T^{\mu\nu}&=F^{\mu\nu}J_\mu,
		\end{align}
		respectively.
		
	\subsection{Hydrodynamic Modes and Linear Response}
	\subsection{Perspetive from Holographic Duality}

\section{Hydrodynamics in Curved Spacetime --- Fractional Quantum Hall System}
	\subsection{Motivation: Hall Viscosity}
		It was observed by Hoyos and Son that Hall viscosity will have a universal impact on electromagnetic response \cite{hoyos2012hall}.
	\subsection{Spacetime Symmetry in FQHE}
		Son then discussed the appropriate spacetime background of low-energy effective theory for nonrelativistic intereacting theory in \cite{son2013newton}. Ward identities are next obtained by Geracie {\it et al.} in \cite{geracie2015spacetime}.

	\subsection{Hydrodynamics on the Lowest Landau Level}
		The main content of this part is to review the work by Son in \cite{geracie2015hydrodynamics}.

\bibliography{hxd}
\bibliographystyle{apsrev} % apsrev is format for PRL of APS
\end{document}